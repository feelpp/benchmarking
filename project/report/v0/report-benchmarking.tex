\documentclass[12pt]{article}

\setlength{\parskip}{\baselineskip}  % ajoute un espace entre les paragraphes

\usepackage{a4wide}
\usepackage[english]{babel}
\usepackage{csquotes}
\usepackage{hyperref}
\usepackage{graphicx}

\usepackage{tocloft}
\renewcommand{\cftsecleader}{\cftdotfill{\cftdotsep}}

\usepackage{biblatex}
\addbibresource{../../bibliography/biblioV0.bib}


\title{\textbf{ExaMA WP3 -- Dashboard Performances}}
\author{Tanguy Pierre\\[0.5cm]
            Supervisors: V. Chabannes, J. Cladellas\\[1cm]
            University of Strasbourg}

\date{26th of March}



\begin{document}
    \maketitle

\newpage
\tableofcontents


\newpage
\section{Introduction}

This work is part of the \textit{ExaMA} project, whose main objective is to design algorithms and methods
that will adapt in the best way to exascale machines and their imminent appearance.\newline
Benchmarking is an unavoidable step for \textit{ExaMA}, as it needs to analyse and compare performances on systems
with different structures. These analyses will not only guide the developer team to improve the scalability of the different available tools,
but also ensure complete transparency during the evaluation process.

As launching tests on supercomputers isn't always easy because of their availability and costs, there is a real need to store the results.
It is exactly what the \textit{Dashboard Performances} project is about: providing a clear and easy to use interface between tests and results.
This interface is already available at \url{https://feelpp.github.io/benchmarking/benchmarking/index.html}.


\section{Tools description}

For testing the implemented algorithms and trying to predict how they will behave, we will use \textit{ReFrame HPC}\cite*{ReFrame}.
ReFrame is a robust framework that allows to focus only on the algorithm by isolating the running code from the system's complexity.
It distinguishes between "performance by simulation step" and "performance by number of task", which are two unavoidable criteria when talking about HPC.
But for us, the most interesting feature is its ability to simulate exascale performance with regressions tests.

For reporting the results, we will use \textit{Antora}\cite*{Antora}. Antora gives the opportunity to publish documentation on the web.
The documentation needs to be written in \textit{AsciiDoc}. Once it's done, \textit{AsciiDoctor} will handle the conversion to \textit{html}
for responsiveness.
In this way, it also call python functions and print their output, graphs for example, like you would do it in a notebook.

\newpage
\section{Objectives}
As the repository has already being started, the necessary resources \textit{(like .adoc templates)} are available
to be able to publish documentation about new tests.

We want to get \textit{Continuous Integration/Continuous Deployment (CI/DI)} system.
It means that every time a new test has been done, and integrated in the directory main branch, a task will 
automatically be launched for updating the documentation site.

In order to achieve this, the first point is to collect all needed information from ReFrame performances and regressions tests
within the created Report python class. Then, it should automatically create an AsciiDoc file which will be used from Antora
for the deployment on the website.

%A first objective would be to work on a first test case and getting all the necessary files ready for publishing
%In a first phase, we will only focus on heat problems.
% A further goal would be to create a whole database


\newpage
\section{Roadmap}
MISSING

\newpage
\section{Bibliography}
\nocite{*}
\printbibliography[heading=none]

\end{document}